\documentclass[11pt]{article}
\usepackage{graphicx}
\graphicspath{ {./images/} }

\usepackage{algorithm}
\usepackage{algpseudocode}
\usepackage{hyperref}

\usepackage{sectsty}
\usepackage{graphicx}
\usepackage[font=small,labelfont=bf]{caption} % Required for specifying captions to tables and figures

% Margins
\topmargin=-0.45in
\evensidemargin=0in
\oddsidemargin=0in
\textwidth=6.5in
\textheight=9.0in
\headsep=0.25in

\title{ %
\includegraphics[width=0.4\textwidth]{UniCT-Logo-Nero}~\\
Trashbin Triplet Classifier \\ 
\large Progetto Deep Learning (LM-18) \\ Università degli Studi di Catania - A.A 2021/2022 \\
}
\author{ Danilo Leocata \\ Docente: Giovanni Maria Farinella, Antonino Furnari}
\date{\today}

\begin{document}

\maketitle	
\pagebreak

%--Paper--

\section{Introduzione}

L'obbiettivo del progetto è realizzare una rete siamese su un dataset contenente secchi della spazzatura.
Per classificare la capienza rimamente di bidoni utilizzati per la spazzatura, e che in particolare sia in grado di distinguere tra: pieno, vuoto, a metà.

È stato già fatto uno studio sul dataset (raccolto) il cui progetto è disponibile al \href{https://github.com/khalld/trashbin-classifier}{seguente} indirizzo.
Il link del progetto attuale è disponibile al seguente indirizzo.


\section{Scelta dell'architettura}

% È stato trovato opportuno l'utilizzo di una \textit{Rete Triplet} per il raggiungimento dell'obbiettivo assegnato, dato che il dataset è composto da 3 classi, è stato pensato che questo approccio sarebbe stato migliore per dare una netta 'differenza' tra la classe 'mezzopieno' e 'vuoto' e 'pieno' (scrivi meglio).
% Ogni tripletta $(I_i, I_j, I_k)$ contiene dunque



\pagebreak

\begin{thebibliography}{5}


\bibitem{1} \href{https://www.researchgate.net/publication/242463011_An_Effective_Algorithm_for_Minimum_Weighted_Vertex_Cover_problem}{An Effective Algorithm for Minimum Weighted Vertex Cover Problem}
\bibitem{2} \href{https://www.sciencedirect.com/science/article/abs/pii/S0377221720300278}{A memory-based iterated local search algorithm for the multi-depot open vehicle routing problem}

\end{thebibliography}


\pagebreak
%--/Paper--

\end{document}
